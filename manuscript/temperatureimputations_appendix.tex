\section{Derivation of the analytic posterior for toy example}
\label{sec:analytical_posterior}

In this appendix we derive and compute the conditional distribution \(\Fcond\) for the toy example of \autoref{sec:toy_example}.
We denote by \(f_i(\cdot)\) and \(F_i(\cdot)\) the prior probability distribution function and cumulative distribution function of \(X_i\), i.e. the normal PDF and CDF with means and variances given by \autoref{eq:toyspec}.
Let \(\pij\) be the probability that \(X_i\) is the minimum of \(X\) and \(X_j\) is its maximum.
We also define \(\pisum = \sum_{j=1}^{100} \pij\), the probability that \(X_i\) is the minimum,
and \(\psumj = \sum_{i=1}^{100} \pij\), the probability that \(X_j\) is the maximum.
The cumulative distribution function of \(X_i\) is then given by:
\begin{equation}
\Pr\del{X_i \leq x \mid \Xmax, \Xmin} =
    \begin{cases}
        0 &\text{if } x < \Xmin \,, \\
        1 &\text{if } x \geq \Xmax \,, \\
        \pxx{i}{\bullet} 
            + \del{1 \!-\! \pxx{i}{\bullet} \!-\! \pxx{\bullet}{i}}
            \sbr{\frac{F_i(x) - F_i(\Xmin) }
                 {F_i(\Xmax) - F_i(\Xmin) }
                } 
            &\text{otherwise.}\\
    \end{cases}
\end{equation}
Meanwhile, \(\pij\) is proportional to:
\begin{equation}
    f_i(\Xmin)
    f_j(\Xmax)
    \prod_{k \neq i,j}^{100}
    \del{F_k(\Xmax) - F_k(\Xmin)} \,,
\end{equation}
which we compute for all \(i,j\) and normalize
to obtain the \(100 \times 100\) matrix \(\Pr\) of probabilities of each pair of element occupying the extremes.
We sum over its rows and columns to obtain \(\psumj\) and \(\pisum\).
While this algorithm has cubic complexity in the dimensionality \(p\) of \(X\),
for \(p=100\), it only take seconds to compute the entries of \(\Pr\) and evaluate \(\Pr\del{X_i \leq x \mid \Xmax, \Xmin}\) over a range of \(x\).
\autoref{fig:toy_quantiles}(b) shows the analytical quantiles of \(\Fcond\) marginally for each \(X_i\).
